\documentclass[openright]{normas-utf-tex} %openright = o capitulo comeca sempre em paginas impares
%\documentclass[oneside]{normas-utf-tex} %oneside = para dissertacoes com numero de paginas menor que 100 (apenas frente da folha) 

% force A4 paper format
\special{papersize=210mm,297mm}

\usepackage[alf,abnt-emphasize=bf,bibjustif,recuo=0cm, abnt-etal-cite=2, abnt-etal-text=it]{abntcite} 

\usepackage{caption} 
\usepackage[font=sf, labelfont=sf, margin=1cm]{caption}

%configuracao correta das referencias bibliograficas.
\usepackage[brazil]{babel} % pacote portugues brasileiro
\usepackage[utf8]{inputenc} % pacote para acentuacao direta
\usepackage{amsmath,amsfonts,amssymb} % pacote matematico
\usepackage{graphicx} % pacote grafico
\usepackage{times} % fonte times
\usepackage[final]{pdfpages} % adicao da ata
\usepackage{verbatim} %fazer blocos comentados
\usepackage{nomencl}
\usepackage{enumerate}
\usepackage{pdfpages}
\makenomenclature
%Podem utilizar GEOMETRY{...} para realizar pequenos ajustes das margens. Onde, left=esquerda, right=direita, top=superior, bottom=inferior. P.ex.:
%\geometry{left=3.0cm,right=1.5cm,top=4cm,bottom=1cm} 
              
% ---------- Preambulo ----------
\instituicao{Universidade ...} %Definir Universidade
\programa{Programa de Pós-graduação em ...} %Definir Pós-graduação

\documento{Relatório de Projeto de Tese em Andamento} %Definir tipo de documento se necessário

\nivel{Doutorado} %Definir Mestrado ou Doutorado
\titulacao{Doutor} %Definir Mestre ou Doutor

\titulo{{TÍTULO DO TRABALHO}}
\title{ÍTULO DO TRABALHO} % titulo do trabalho em ingles

\autor{Fulano DETAL} 
\cita{DETAL, Fulano} 

%\palavraschave{Palavra-chave 1, Palavra-chave 2, ...} % palavras-chave do trabalho
%\keywords{Keyword 1, Keyword 2, ...} % palavras-chave do trabalho em ingles

\comentario{Tipo do documento apresentado ao tal Programa de Pós-graduação da Universidade TAL como requisito parcial para obtenção do grau de "Doutor/Mestre em TAL ÁREA" -- Área de Concentração: Qualquer.}

\orientador{Prof. Dr. Beltrano DETAL}
\coorientador{Prof. Dr. Cicrano QUALQUER} 

\local{Cidade} 
\data{\the\year} % ano automatico

% desativa hifenizacao mantendo o texto justificado.
\tolerance=1
\emergencystretch=\maxdimen
\hyphenpenalty=10000
\hbadness=10000
\sloppy

%---------- Inicio do Documento ----------
\begin{document}

% geracao automatica da capa
\capa 
% geracao automatica da folha de rosto
\folhaderosto 

%Resumo
\begin{resumo}
Resumo do trabalho
\end{resumo}

\begin{abstract}
Abstract of document
\end{abstract}

\listadefiguras %geração de lista de figuras
\listadesiglas %geração de lista de siglas
\printnomenclature %manter para geração das listas


\sumario % sumario

%---------- Inicio do Texto ----------


%---------- Primeiro Capitulo ----------
\chapter{Introdução}

\section{Título da seção}
Iniciar a escrita de sua introdução considerando as citações presentes no arquivo.bib de seu projeto \cite{Hagen2000TheBioinformatics}. Importante lembrar que todas as referências utilizadas tem que estra presentes no arquivo.bib, caso contrário seu projeto não irá compilar \cite{Hillmer2015SystemsBiologists}. 

\section{Título da seção}
Se for necessário usar alguma palavra em itálico usar \textit{palavra em ítálico}. A inserção de siglas no texto ocorre dessa maneira \sigla{SEPT}{Setor de Educação Profissional e Tecnológica} e assim ela tbm vai automaticamente para a lista de siglas.
Outra seção e assim por diante. Veja abaixo o exemplo para inserção de figura. Esse exemplo está comentado mas é só tirar do comentário depois de fazer o input da  (\textbf{figuraexemplo}).


%\begin{figure}[!htb] 
%\centering
%\includegraphics[width=0.8\textwidth]{./figuraexemplo.eps}
%\caption[Título da figura]{\textbf{Título da figura}.Descrição detalhada da figura (fonte, ano).}
%\label{fig:tal}
%\end{figure}

 



\chapter{Objetivos}

\section{Objetivo Geral}

Descrever. 

\section{Objetivos Específicos}

\begin{itemize}
	\item Descrever. 

	\item Descrever. 

	\item Descrever. 
\end{itemize}


\chapter{Justificativa}

Descrever.



\chapter{Metodologia}

\section{Dados de parara}
Descrever.


\chapter{Resultados preliminares e discussão}

Descrever.



 
\chapter{Conclusões}

Descrever.
%---------- Referencias ----------
\clearpage % this is need for add +1 to pageref of bibstart used in 'ficha catalografica'.
\label{bibstart}
\bibliography{Bibliografia} % geracao automatica das referencias a partir do arquivo Bibliografia.bib
\label{bibend}

% ---------- Anexos (opcionais) ----------

\anexo

\chapter{Lista de produções}

\begin{enumerate}
\begin{large}
\item Nome do Ítem de inserção (\textbf{anexo B})
\end{large}

DETAL1, F.T.B.; DETAL2, Q.; DETAL3, S.N.P.; DETAL4, S.P.A.
Título da Produção \\
DOI: https://doi.org/doi:XX.XXXXX/XX.aaa.bbbbb

\end{enumerate}

%%%%%%%%%%%%%%%%%%%%%%%%%%%%%%%%%%
\chapter{Pacote de dados do \textit{Ferrament}}
%\begin{figure}[!htb]                
%\centering                   
%\includegraphics[width = 1.00\textwidth]{./figura.eps}                      
%    \footnotesize Fonte: https://www.parara.org     
%\end{figure}

%%%%%%%%%%%%%%%%%%%%%%%%%%%%%%%%%%

% --------- Ordenacao Afabetica da Lista de siglas --------
%\textbf{* Observações:} a ordenacao alfabetica da lista de siglas ainda nao eh realizada de forma automatica, porem é possivel de realizar isto manualmente. Duas formas:
%
% ** Primeira forma)
%    A ordenacao eh feita com o auxilio do comando 'sort', disponivel em qualquer
% sistema Linux e UNIX, e tambem em sistemas Windows se instalado o coreutils (http://gnuwin32.sourceforge.net/packages/coreutils.htm)
% comandos para compilar e ordenar, supondo que seu arquivo se chame 'dissertacao.tex':
%
%      $ latex dissertacao
%      $ bibtex dissertacao && latex dissertacao
%      $ latex dissertacao
%      $ sort dissertacao.lsg > dissertacao.lsg.tmp
%      $ mv dissertacao.lsg.tmp dissertacao.lsg
%      $ latex dissertacao
%      $ dvipdf dissertacao.dvi
%
%
% ** Segunda forma)
%\textbf{Sugest\~ao:} crie outro arquivo .tex para siglas e utilize o comando \sigla{sigla}{descri\c{c}\~ao}.
%Para incluir este arquivo no final do arquivo, utilize o comando \input{arquivo.tex}.
%Assim, Todas as siglas serao geradas na ultima pagina. Entao, devera excluir a ultima pagina da versao final do arquivo
% PDF do seu documento.

%-------- Citacoes ---------
% - Utilize o comando \citeonline{...} para citacoes com o seguinte formato: Autor et al. (2011).
% Este tipo de formato eh utilizado no comeco do paragrafo. P.ex.: \citeonline{autor2011}
% - Utilize o comando \cite{...} para citações no meio ou final do paragrafo. P.ex.: \cite{autor2011}

\end{document}

